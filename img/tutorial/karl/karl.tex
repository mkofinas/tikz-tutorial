\begin{tikzpicture}[
  scale=3, line cap=round,
  % Styles
  axes/.style=,
  important line/.style={very thick},
  information text/.style={rounded corners, fill=red!10, inner sep=1ex}
]
  % Colors
  \colorlet{anglecolor}{green!50!black}
  \colorlet{sincolor}{red}
  \colorlet{tancolor}{orange!80!black}
  \colorlet{coscolor}{blue}
 
  % The graphic
  \draw[help lines,step=0.5cm] (-1.4,-1.4) grid (1.4,1.4);
  \draw (0,0) circle [radius=1cm];
  \begin{scope}[axes]
    \draw[->] (-1.5,0) -- (1.5,0) node[right] {$x$} coordinate(x axis);
    \draw[->] (0,-1.5) -- (0,1.5) node[above] {$y$} coordinate(y axis);
    \foreach \x/\xtext in {-1, -.5/-\frac{1}{2}, 1}
      \draw[xshift=\x cm] (0pt,1pt) -- (0pt,-1pt) node[below,fill=white] {$\xtext$};
    \foreach \y/\ytext in {-1, -.5/-\frac{1}{2}, .5/\frac{1}{2}, 1}
      \draw[yshift=\y cm] (1pt,0pt) -- (-1pt,0pt) node[left,fill=white] {$\ytext$};
  \end{scope}
 
  \filldraw[fill=green!20,draw=anglecolor] (0,0) -- (3mm,0pt) arc [start angle=0, end angle=30, radius=3mm];
  \draw (15:2mm) node[anglecolor] {$\alpha$};
  \draw[important line,sincolor] (30:1cm) -- node[left=1pt,fill=white] {$\sin \alpha$} (30:1cm |- x axis);
  \draw[important line,coscolor] (30:1cm |- x axis) -- node[below=2pt,fill=white] {$\cos \alpha$} (0,0);
  \path [name path=upward line] (1,0) -- (1,1);
  \path [name path=sloped line] (0,0) -- (30:1.5cm);
  \draw [name intersections={of=upward line and sloped line, by=t}]
    [very thick,orange] (1,0) -- node [right=1pt,fill=white]
    {$\displaystyle \tan \alpha \color{black}=\frac{{\color{sincolor}\sin \alpha}}{\color{coscolor}\cos \alpha}$} (t);
  \draw (0,0) -- (t);
  \draw[xshift=1.85cm] node[right,text width=6cm,information text]
  {The {\color{anglecolor} angle $\alpha$} is $30^\circ$ in the example ($\pi/6$ in radians). 
   The {\color{sincolor}sine of $\alpha$}, which is the height of the red line, is
   \[{\color{sincolor} \sin \alpha} = 1/2.\]
   By the Theorem of Pythagoras we have ${\color{coscolor}\cos^2 \alpha} + {\color{sincolor}\sin^2 \alpha} = 1$.
   Thus the length of the blue line, which is the {\color{coscolor} cosine of $\alpha$}, must be
   \[{\color{coscolor}\cos \alpha} = \sqrt{1 - 1/4} = \frac{1}{2}\sqrt{3}.\]
   This shows that ${\color{tancolor}\tan \alpha}$, which is the height of the orange line, is
   \[{\color{tancolor}\tan \alpha} = \frac{{\color{sincolor}\sin \alpha}}{{\color{coscolor}\cos \alpha}} = 1/\sqrt{3}.\]
  };
\end{tikzpicture}